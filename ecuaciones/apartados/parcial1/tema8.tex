\chapter{Ecuaciones Homógeneas}
Una función \(\displaystyle f(x,y)\) es homógenea de grado \(\displaystyle n\), si en sus argumentos si cumple lo siguiente:

\begin{equation*}
    \begin{gathered}
        f(tx,ty)=t^{n}f(x,y)
    \end{gathered}
\end{equation*}

\textbf{Ejemplo}\\

Sea \(\displaystyle f(x,y)=x^{2}+y^{2}-2xy\), verificar que es una función de grado \(\displaystyle 2\):\\

\textit{Solución}

\begin{equation*}
    \begin{gathered}
        f(tx,ty)=t^{2}x^{2}+t^{2}y^{2}-2txty\\\\
        =t^{2}x^{2}+t^{2}y^{2}-2t^{2}xy\\\\
        =t^{2}(x^{2}+y^{2}-2xy)\\\\
        \Rightarrow\;\;\;t^{2}f(x,y)
    \end{gathered}
\end{equation*}
Para \(\displaystyle n=0\), se tiene una función de grado \(\displaystyle 0\)\\

\textbf{Ejemplo}\\

Sea \(\displaystyle f(x,y)=\frac{x^{2}-y^{2}}{x^{2}+y^{2}}\)

\textit{Solución}

\begin{equation*}
    \begin{gathered}
        f(tx,ty)=\frac{t^{2}x^{2}-t^{2}y^{2}}{t^{2}x^{2}+t^{2}y^{2}}\;\;\;\Leftrightarrow\;\;\;f(tx,ty)=\frac{t^{2}}{t^{2}}\frac{x^{2}-y^{2}}{x^{2}+y^{2}}\\\\
        \therefore t^{0}
    \end{gathered}
\end{equation*}

\section{Ecuaciones Homógeneas en ED}

Para resolver ED es necesario plantear un modelo de cambio de variable, el cual te permitira resolver la ecuación diferencial de grado \(\displaystyle n\)\\

Resolver las siguientes ED's homógeneas:

\begin{equation}
    \begin{gathered}
        xy'=\sqrt{x^{2}-y^{2}}+y
    \end{gathered}
\end{equation}

\textit{Solución}
\begin{equation*}
    \begin{gathered}
        y'=\frac{\sqrt{x^{2}-y^{2}}+y}{x}\\\\
        y'=\sqrt{\frac{x^{2}}{x^{2}}-\frac{y^{2}}{x^{2}}}+\frac{y}{x}
        \Leftrightarrow y'=\sqrt{1-\frac{y^{2}}{x^{2}}}+\frac{y}{x}\rightarrow (1)\\\\
        f(x,y)=\sqrt{1-\frac{y^{2}}{x^{2}}}+\frac{y}{x}\\\\
        f(tx,ty)=\sqrt{1-\left(\frac{ty}{tx}\right)^{2}}+\frac{ty}{tx}\\\\
        \therefore\;\;Es\;\;Hom\acute{o}genea
    \end{gathered}
\end{equation*}

Por lo tanto propondremos el siguiente cambio de variable:

\begin{equation*}
    \begin{gathered}
        u=\frac{y}{x}\;\;\;\;\;\;\;\;\;\;\;\;y=ux\;\;\;\;\Rightarrow\;\;\;\;y'=u+x\frac{du}{dx}\rightarrow(2)
    \end{gathered}
\end{equation*}

Ahora sustituimos \(\displaystyle (2)\) en \(\displaystyle (1)\):

\begin{equation*}
    \begin{gathered}
        u+xu'=\sqrt{1-u^{2}}+u\\\\
        xu'=\sqrt{1-u^{2}}\\\\
        \Leftrightarrow\;\;\;\;x\frac{du}{dx}=\sqrt{1-u^{2}}
    \end{gathered}
\end{equation*}

Ahora solo procederemos a resolver la ED, no olvidando el cambio de variable:

\begin{equation*}
    \begin{gathered}
        xdu=\sqrt{1-u^{2}}dx\\\\
        \Leftrightarrow\;\;\;\;\frac{du}{\sqrt{1-u^{2}}}=\frac{dx}{x}\\\\
        \int\frac{du}{\sqrt{1-u^{2}}}=\int\frac{dx}{x}
    \end{gathered}
\end{equation*}

Resolviendo la integral tendremos:

\begin{equation*}
    \begin{gathered}
        \arcsin(u)=\ln(x)+C\;\;\;\;\Leftrightarrow\;\;\;\;u=\sin\left(\ln\left|x\right|+C\right)
    \end{gathered}
\end{equation*}

Sustituyendo el valor de \(\displaystyle u\)

\begin{equation*}
    \begin{gathered}
        \frac{y}{x}=\sin\left(\ln\left|x\right|+C\right)\;\;\;\;\Leftrightarrow\;\;\;\;y=x\sin\left(\ln\left|x\right|+C\right)
    \end{gathered}
\end{equation*}

\clearpage

\begin{equation}
    \begin{gathered}
        (x-y)dx+(x+y)dy=0
    \end{gathered}
\end{equation}

\textit{Solución}

\begin{equation*}
    \begin{gathered}
        (x-y)dx+(x+y)dy=0\;\;\;\;\Leftrightarrow\;\;\;\;(x+y)dy=(y-x)dx\\\\
        \frac{dy}{dx}=\frac{(y-x)}{(x+y)}\rightarrow(1)\;\;\;\;\Rightarrow\;\;\;\;f(x,y)=\frac{(y-x)}{(x+y)}
    \end{gathered}
\end{equation*}

Entonces comprobaremos si \(\displaystyle f(x,y)\) es homógenea:

\begin{equation*}
    \begin{gathered}
        f(tx,ty)=\frac{ty-tx}{tx+ty}\\\\
        =\frac{t}{t}\frac{y-x}{x+y}\\\\
        \therefore\;Es\;\;hom\acute{o}genea\\\\
        f(x,y)=\frac{x}{x}\frac{\cfrac{y}{x}-1}{1+\cfrac{y}{x}}\;\;\;\;=\;\;\;\;\frac{\cfrac{y}{x}-1}{1+\cfrac{y}{x}}
    \end{gathered}
\end{equation*}

Entonces con el cambio de variable propondremos:

\begin{equation*}
    \begin{gathered}
        u=\frac{y}{x}\;\;\;\;\;\;\;\;\;\;\;\;y=ux\;\;\;\;\;\;\;\;\;\;\;\;y'=u+x\frac{du}{dx}\rightarrow(2)
    \end{gathered}
\end{equation*}

Ahora sustituyendo \(\displaystyle (2)\) en \(\displaystyle (1)\):

\begin{equation*}
    \begin{gathered}
        u+x\frac{du}{dx}=\frac{u-1}{u+1}\;\;\;\;\Leftrightarrow \;\;\;\;x\frac{du}{dx}=\frac{u-1}{u+1}-u\\\\
        x\frac{du}{dx}=\frac{u-1-u(u+1)}{u+1}\;\;\;\;\Leftrightarrow \;\;\;\;x\frac{du}{dx}=\frac{u-1-u^{2}-u}{u+1}\\\\
        x\frac{du}{dx}=-\frac{1+u^{2}}{u+1}\;\;\;\;\Leftrightarrow \;\;\;\;\frac{u+1}{u^{2}+1}du=-\frac{dx}{x}\\\\
        \int\frac{u+1}{u^{2}+1}du=-\int\frac{dx}{x}\\\\
        \int\frac{u}{u^{2}+1}du+\int\frac{du}{u^{2}+1}=-\int\frac{dx}{x}
    \end{gathered}
\end{equation*}

Resolviendo cada integral tendremos los siguientes resultados:

\begin{equation*}
    \begin{gathered}
        \frac{1}{2}\ln\left|u^{2}+1\right|+\arctan(u)=-\ln\left|x\right|+C\\\\
        \frac{1}{2}\ln\left|u^{2}+1\right|+\arctan(u)+\ln\left|x\right|=C\\\\
    \end{gathered}
\end{equation*}

Aplicando leyes de los logaritmos:

\begin{equation*}
    \begin{gathered}
        \ln\left|x\left(u^{2}+1\right)^{\frac{1}{2}}\right|+\arctan(u)=C
    \end{gathered}
\end{equation*}

Sustituyendo los valores de \(\displaystyle u\)

\begin{equation*}
    \begin{gathered}
        \ln\left|x\left(\left(\frac{y}{x}\right)^{2}+1\right)^{\frac{1}{2}}\right|+\arctan\left(\frac{y}{x}\right)=C
    \end{gathered}
\end{equation*}

\vspace{1cm}
\begin{equation}
    \begin{gathered}
        ydx+x\left(\ln(x)-\ln(y)-1\right)dy=0;\;\;\;\;y(1)=e
    \end{gathered}
\end{equation}

\textit{Solución}

\begin{equation*}
    \begin{gathered}
        ydx+x(\ln(x)-\ln(y)-1)dy=0\;\;\;\;\Leftrightarrow\;\;\;\;ydx+x\left\{\ln(x)-\right[\ln(y)+\ln(e)\left]\right\}dy=0\\\\
        ydx+x\left[\ln(x)-\ln(ey)\right]dy=0\;\;\;\;\Leftrightarrow\;\;\;\;ydx+x\ln\left(\frac{x}{ey}\right)dy=0\\\\
        x\ln\left(\frac{x}{ey}\right)dy=-ydx\;\;\;\;\Leftrightarrow\;\;\;\;x\ln\left(\frac{x}{ey}\right)\frac{dy}{dx}=-y\\\\
        \frac{dy}{dx}=-\frac{y}{x\ln\left(\frac{x}{ey}\right)}\;\;\;\;\Leftrightarrow\;\;\;\;\frac{dy}{dx}=\frac{y}{x\ln\left(\frac{ey}{x}\right)}
    \end{gathered}
\end{equation*}

Reescribiremos lo siguiente:

\begin{equation*}
    \begin{gathered}
        f(x,y)=\frac{y}{x\ln\left(\frac{ey}{x}\right)}\\\\
        f(tx,ty)=\frac{t}{t}\frac{y}{x\ln\left(\frac{t}{t}\frac{ey}{x}\right)}\;\;\;\;\therefore\;\;Es\;\;hom\acute{o}genea
    \end{gathered}
\end{equation*}

Proponiendo el cambio de variable:

\begin{equation*}
    \begin{gathered}
        u=\frac{y}{x};\;\;\;\;y=ux;\;\;\;\;\frac{dy}{dx}=u+x\frac{du}{dx}
    \end{gathered}
\end{equation*}

Entonces:
\begin{equation*}
    \begin{gathered}
        u+x\frac{du}{dx}=\frac{u}{\ln(eu)}\;\;\;\;\Leftrightarrow\;\;\;\;x\frac{du}{dx}=\frac{u-u\ln(eu)}{\ln(eu)}\\\\
        du=\frac{1}{x}\frac{u-u\ln(eu)}{\ln(eu)}dx\;\;\;\;\Leftrightarrow\;\;\;\;\frac{\ln(eu)}{u-u\ln(eu)}du=\frac{dx}{x}
    \end{gathered}
\end{equation*}

Resolviendo la integral:

\begin{equation*}
    \begin{gathered}
        \int\frac{\ln(eu)}{u-u\ln(eu)}du=\int\frac{dx}{x}
    \end{gathered}
\end{equation*}

Para resolver, podemos proponer el cambio de variable, el cual quedaria con lo siguiente

\begin{equation*}
    \begin{gathered}
        v=u-u\ln(eu)\;\;\;\;\Rightarrow\;\;\;\;dv=1-\ln(eu)+1\;\;\;\therefore\;dv=-\ln(eu)
    \end{gathered}
\end{equation*}

Ahora:

\begin{equation*}
    \begin{gathered}
        -\int\frac{dv}{v}=\int\frac{dx}{x}\;\;\;\;\Leftrightarrow\;\;\;\;-\ln|v|=\ln|x|+C\\\\
        -\ln|u-u\ln(eu)|=\ln|x|+C\;\;\;\;\Leftrightarrow\;\;\;\;-\ln\left|\frac{y}{x}-\frac{y}{x}\ln\left(\frac{ey}{x}\right)\right|=\ln|x|+C
    \end{gathered}
\end{equation*}

Solución general:

\begin{equation*}
    \begin{gathered}
        \ln\left|\frac{y}{x}-\frac{y}{x}\ln\left(\frac{ey}{x}\right)\right|+\ln|x|+C=0
    \end{gathered}
\end{equation*}

Solución particular \(\displaystyle y(1)=e\):

\begin{equation*}
    \begin{gathered}
        \ln\left|e-e\ln(e)^{2}\right|+C=0\;\;\;\;\Leftrightarrow\;\;\;\;\ln\left|e-2e\ln(e)\right|+C=0
    \end{gathered}
\end{equation*}

Donde \(\displaystyle \ln\left|e\right|=\ln\left|-e\right|=1\)

\begin{equation*}
    \begin{gathered}
        \ln|e-2e|+C=0\;\;\;\;\Leftrightarrow\;\;\;\;\ln|-e|+C=0\;\;\;\therefore\;\;C=-1
    \end{gathered}
\end{equation*}

Solución

\begin{equation*}
    \begin{gathered}
        \ln\left|\frac{y}{x}-\frac{y}{x}\ln\left(\frac{ey}{x}\right)\right|+\ln|x|-1=0
    \end{gathered}
\end{equation*}
\clearpage
