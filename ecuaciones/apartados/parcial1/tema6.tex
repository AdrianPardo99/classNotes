\chapter{Soluciones}

\section{Solución de una ED}

Uno de los objetivos primordiales del curso es resolver o encontrar soluciones de EDO's.\\

\textbf{Definición:} Cualquier función \(\displaystyle\phi(x)\), definida en un intervalo \(\displaystyle I \) y con al menos n-derivadas continuas en dicho intervalo de \(\displaystyle I\), que al sustituirse en una EDO de n-esimo orden reduce a la ED en una identidad, se considera solución de dicha EDO.\\

\textbf{Ejemplos:}

\begin{equation}
    \begin{gathered}
        a)\;\frac{dy}{dx}=xy^{\frac{1}{2}}\;\;;\;\;y=\frac{x^{4}}{16}\\\\
        b)\;y''-2y'+y=0\;\;;\;\;y=xe^{x}
    \end{gathered}
\end{equation}

\textit{Solución} \(\displaystyle 4.1\), la cual derivaremos y sustituiremos los valores en las ecuaciones correspodientes, se procedera directamente:

\begin{equation*}
    \begin{gathered}
        a)\;\frac{x^{3}}{4}=x\left(\frac{x^{2}}{4}\right)\\\\
        b)\; (x+2)e^{x}-2[(x+1)e^{x}]+xe^{x}=0
    \end{gathered}
\end{equation*}

Finalmente resolviendo estas ecuaciones algebraicas podemos probar que la ED es correcta en primer instancia.\\

\textit{Nota:} Una solución de una ED que es identicamente a cero en un intervalo \(\displaystyle I\) se llama solución trivial (Si \(\displaystyle y\equiv0\))

\section{Solución explicita e implicita}

\textbf{Definición solución explicita:} Se dice que una solución es explicita, si en dicha solución la variable dependiente se expresa únicamente en términos de la variable independiente y constantes\\

\textbf{Ejemplos:}

\begin{equation*}
    \begin{gathered}
        y(x)=\frac{x^{2}}{2}+4\;\;\;\;\;\;\;\;\;\;\;\;y(x)=9x^{2}+5x+e^{5x}
    \end{gathered}
\end{equation*}


\textbf{Definición solución implicita} Una relación \(\displaystyle G(x,y)=0\) es una solución implicita de una EDO en un intervalo \(\displaystyle I\), siempre que exista al menos una función \(\displaystyle\phi(x)\) que satisfaga tanto a la relación como  a la ED en \(\displaystyle I\)\\

\textbf{Ejemplo}

La relación \(\displaystyle x^{2}+y^{2}=25\) es una solución implicita de la ED

\begin{equation}
    \begin{gathered}
        \frac{dy}{dx}=-\frac{x}{y}\;\;\;\;\;\;\;en\;\;\;I=(-5,5)
    \end{gathered}
\end{equation}

\textit{Solución}\\

Derivamos implicitamente a la relación:

\begin{equation*}
    \begin{gathered}
        2x+2yy'=0 \;\Rightarrow\; 2yy'=-2x \\\\
        \Rightarrow\; y'=-\frac{2x}{2y}\\\\
        \Rightarrow\; y'=-\frac{x}{y}
    \end{gathered}
\end{equation*}

Por otra parte:

\begin{equation*}
    \begin{gathered}
        \left|y\right|=\sqrt{25-x^{2}}\\\\
        \Rightarrow\;\phi_{1}(x)=\sqrt{25-x^{2}}\;\;\;\&\;\;\;\phi_{2}(x)=-\sqrt{25-x^{2}}
    \end{gathered}
\end{equation*}

\section{Solución general, condiciones inciales; Solución particular y unicidad de la solución}

Para que sea más ilustrativo este tema se mostrara el como se realiza a traves de una ecuación.\\

Sea \(\displaystyle\frac{dy}{dx}=2x\), encontrar su solución general y su solución particular si \(\displaystyle y(x=0)=1\)\\

\textit{Solución}

\begin{equation}
    \begin{gathered}
        \frac{dy}{dx}=2x\:\:\;\Rightarrow dy=2xdx\:\:\;\Rightarrow \int dy=\int 2xdx
    \end{gathered}
\end{equation}

En su solución general es decir que se tiene una familia de curvas bajo la constante \(\displaystyle C\) de la integral obtendremos:

\begin{equation*}
    \begin{gathered}
        y(x)=x^{2}+C\;\;\;|\;C \in \mathbb{R}\;\;\;\;(Sol.\;\;general)
    \end{gathered}
\end{equation*}

Para su solución particular usaremos la condición inicial dada al inicio, obteniendo lo siguiente con \(\displaystyle x=0\):

\begin{equation*}
    \begin{gathered}
        y(0)\equiv 1 \\\\
        1=(0)^{2}+C \Rightarrow C=1\\\\
        \therefore y(x)=x^{2}+1\;\;\;\;(Sol.\;\;particular)
    \end{gathered}
\end{equation*}