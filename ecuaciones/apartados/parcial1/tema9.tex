\chapter{Ecuaciones Exactas}

Sea \(\displaystyle z=f(x,y)\) una función de dos variables con primeras derivadas parciales continuas en una región \(\displaystyle R\tiny{xy}\) del plano \(\displaystyle xy\), entonces su diferencial es:

\begin{equation}
    \begin{gathered}
        dz=\frac{\partial{f(x,y)}}{\partial{x}}dx+\frac{\partial{f(x,y)}}{\partial{y}}dy
    \end{gathered}
\end{equation}

Si \(\displaystyle f(x,y)=C;\;\;\;C=cte\), entonces \(\displaystyle (7.1)\) se transforma en:

\begin{equation}
    \begin{gathered}
        \frac{\partial{f(x,y)}}{\partial{x}}dx+\frac{\partial{f(x,y)}}{\partial{y}}dy=0
    \end{gathered}
\end{equation}
Dada una familia de funciones \(\displaystyle f(x,y)=C\) puede generar una ED de primer orden calculando la diferencial en ambos lados de la igualdad:\\

\textbf{Ejemplo}\\

Si \(\displaystyle x^{2}-5xy+y^{2}=C\) el inciso \(\displaystyle (7.2)\) provee la ED de primer orden

\begin{equation*}
    \begin{gathered}
        (2x-5y)dx+(-5x+2y)dy=0
    \end{gathered}
\end{equation*}
De tal forma que la ED es de la forma:
\begin{equation}
    \begin{gathered}
        M(x,y)dx+N(x,y)dy=0
    \end{gathered}
\end{equation}
Se llama ED exacta si su primer miembro es la diferencial total de una función \(\displaystyle f(x.y)\), es decir:

\begin{equation*}
    \begin{gathered}
        M(x,y)dx+N(x,y)dy=df(x,y)=\frac{\partial{f(x,y)}}{\partial{x}}dx+\frac{\partial{f(x,y)}}{\partial{y}}dy
    \end{gathered}
\end{equation*}

Entonces obtener la solución de una ED exacta consiste en encontrar una función \(\displaystyle f(x,y)=C\), tal que su diferencial sea exactamente la ED que se pretende resolver.\\

\textbf{Ejemplo}\\

Supongase que se tiene la ED:

\begin{equation*}
    \begin{gathered}
        2xydx+x^{2}dy=9
    \end{gathered}
\end{equation*}

Que es exacta ya que proviene de la función:

\begin{equation*}
    \begin{gathered}
        z=f(x,y)=x^{2}y
    \end{gathered}
\end{equation*}

Donde podemos observar que:

\begin{equation*}
    \begin{gathered}
        \frac{\partial{f}}{\partial{x}}=2xy;\;\;\;\frac{\partial{f}}{\partial{y}}=x^{2}\\\\
        \Rightarrow\;\frac{\partial{f}}{\partial{x}}dx+\frac{\partial{f}}{\partial{y}}dy=2xydx+x^{2}dy=0\\\\
    \end{gathered}
\end{equation*}

\begin{equation}
    \begin{gathered}
        \Rightarrow\;\;M(x,y)=\frac{\partial{f}}{\partial{x}}
    \end{gathered}
\end{equation}

\begin{equation}
    \begin{gathered}
        \Rightarrow\;\;N(x,y)=\frac{\partial{f}}{\partial{y}}
    \end{gathered}
\end{equation}

Ahora, si derivamos a la ecuaciones \(\displaystyle (7.3)\) y \(\displaystyle (7.4)\), pero ahora con respecto a la otra variable respectivamente:

\begin{equation}
    \begin{gathered}
        \frac{\partial{M}}{\partial{y}}=\frac{\partial}{\partial{y}}\frac{\partial{M}}{\partial{x}}=\frac{\partial^{2}{M}}{\partial{y}\partial{x}}
    \end{gathered}
\end{equation}

\begin{equation}
    \begin{gathered}
        \frac{\partial{N}}{\partial{x}}=\frac{\partial}{\partial{x}}\frac{\partial{M}}{\partial{y}}=\frac{\partial^{2}{N}}{\partial{x}\partial{y}}
    \end{gathered}
\end{equation}

Continuando, como supusimos desde un inicio que las derivadas parciales de \(\displaystyle f(x,y)\) son continuas, tendremos:

\begin{equation}
    \begin{gathered}
        \frac{\partial{M}}{\partial{y}}\equiv \frac{\partial{N}}{\partial{x}}\;\;Condici\acute{o}n\;para\;que\;una\;ED\;sea\;exacta
    \end{gathered}
\end{equation}

\textit{Nota:} Por tanto la ED es exacta si se cumple la condicion \(\displaystyle (7.8)\)\\

Para poder resolver una ED exacta recurriremos a las ecuaciones \(\displaystyle (7.4)\) o \(\displaystyle (7.5)\), es decir, nos apoyaremos de dichas expresiones:

\begin{equation}
    \begin{gathered}
        M(x,y)=\frac{\partial{f}}{\partial{x}}
    \end{gathered}
\end{equation}
\begin{equation}
    \begin{gathered}
        N(x,y)=\frac{\partial{f}}{\partial{y}}
    \end{gathered}
\end{equation}
Entonces si tomamos la ecuación \(\displaystyle (7.9)\) e integramos parcialmente respecto a \(\displaystyle x\) obtenemos \(\displaystyle f(x,y)\), de tal forma que:
\begin{equation}
    \begin{gathered}
        f(x,y)=\int M(x,y)dx+h(y)
    \end{gathered}
\end{equation}

Que es la solución buscada, solo falta hallar o determinar \(\displaystyle h(y)\) que es una expresion que representa a la constante arbitraría de la integral anterior.\\

Para determinar \(\displaystyle h(y)\) se deriva a la \(\displaystyle (xi)\) respecto de \(\displaystyle y\) y se obtiene:

\begin{equation}
    \begin{gathered}
        \frac{\partial{f}}{\partial{y}}=N(x,y)=\frac{\partial}{\partial{y}}\int M(x,y)dx + h'(y)
    \end{gathered}
\end{equation}

Que al despejar se tiene:

\begin{equation}
    \begin{gathered}
        h'(y)=N(x,y)=-\frac{\partial}{\partial{y}}\int M(x,y)dx
    \end{gathered}
\end{equation}
Y finalmente para conocer \(\displaystyle h(y)\) se integra a \(\displaystyle (7.13)\).\\

\textit{Nota: }Resaltemos que para el caso de integrar con respecto a \(\displaystyle y\) es el mismo procedimiento, solo cambia la variable con la que se trabaja

\clearpage

\textbf{Ejemplo}

\begin{equation}
    \begin{gathered}
        2y\sin(x)\cos(x)dx+\sin^{2}(x)dy=0
    \end{gathered}
\end{equation}

\textit{Solución}\\

Paso 1: Determinar si la ecuación es exacta:

\begin{equation*}
    \begin{gathered}
        M(x,y)=2y\sin(x)\cos(x);\;\;\;\;N(x,y)=\sin^{2}(x)\\\\
        \Rightarrow \frac{\partial{M}}{\partial{y}}=2\sin(x)\cos(x)\;\;\;\&\;\;\;\frac{\partial{N}}{\partial{x}}=2\sin(x)\cos(x)\\\
        \therefore\;\;\frac{\partial{M}}{\partial{y}}=\frac{\partial{N}}{\partial{x}}\;\;La\;ED\;es\;exacta
    \end{gathered}
\end{equation*}

Paso 2: seleccionar que vamos a integrar (en este caso sobre \(\displaystyle x\))

\begin{equation*}
    \begin{gathered}
        \frac{\partial{f}}{\partial{x}}=M(x,y)\\\\
        \int\frac{\partial{f}}{\partial{x}}=\int M(x,y)\;\;\;\;\Leftrightarrow\;\;\;\;f(x,y)=2y\int\sin(x)\cos(x)dx
    \end{gathered}
\end{equation*}

Paso 3: Resoviendo la integral:

\begin{equation*}
    \begin{gathered}
        f(x,y)=y\sin^{2}(x)+h(y)\;\;Soluci\acute{o}n\;general
    \end{gathered}
\end{equation*}

Paso 4: Ahora derivando con respecto a \(\displaystyle y\)

\begin{equation*}
    \begin{gathered}
        \frac{\partial{f}}{\partial{y}}=\sin^{2}(x)+h'(y)\equiv N(x,y)=\sin^{2}(x)\\\\
        \Rightarrow\;\;h'(y)=0\;\;\;\;\int h'(y)=\int 0\\\\
        \therefore\;\;h(y)=C\\\\
        f(x,y)=y\sin^{2}(x)+C\;\;Soluci\acute{o}n\;general
    \end{gathered}
\end{equation*}

Ahora pensando en que \(\displaystyle y(x)\)

\begin{equation*}
    \begin{gathered}
        y(x)=\frac{C}{\sin^{2}(x)}
    \end{gathered}
\end{equation*}
\clearpage
Algunos otros ejemplos son:

\begin{equation}
    \begin{gathered}
        (ye^{xy}+2x-1)dx+(xe^{xy}-2y+1)dy=0
    \end{gathered}
\end{equation}

\textit{Solución}

\begin{equation*}
    \begin{gathered}
        M(x,y)=ye^{xy}+2x-1\;\;\;\&\;\;\;N(x,y)=xe^{xy}-2y+1\\\\
        \frac{\partial{M}}{\partial{y}}=xye^{xy}+e^{xy}\;\;\;\&\;\;\;\frac{\partial{N}}{\partial{x}}=xye^{xy}+e^{xy}\\\\
        \therefore\;\;\frac{\partial{M}}{\partial{y}}\equiv\frac{\partial{N}}{\partial{x}}\\\\
        \int\frac{\partial{f}}{\partial{x}}dx=\int M(x,y)dx\\\\
        f(x,y)=e^{xy}+x^{2}-x+h(y)
    \end{gathered}
\end{equation*}

Ahora para encontrar \(\displaystyle h(y)\)

\begin{equation*}
    \begin{gathered}
        \frac{\partial{f}}{\partial{y}}=xe^{xy}+h'(y)\equiv N(x,y)=xe^{xy}-2y+1\\\\
        h'(y)=1-2y\;\;\;\Rightarrow\;\;\;\int h'(y)dy=\int(1-2y)dy\\\\
        h(y)=y-y^{2}+C\\\\
        f(x,y)=e^{xy}+x^{2}-x-y^{2}+y+C\\\\
        f(x,y)=e^{xy}+x^{2}-x-y^{2}+y=C
    \end{gathered}
\end{equation*}

\vspace{1cm}

\begin{equation}
    \begin{gathered}
        (2x+6y)dx+(2y-6x)dy=0
    \end{gathered}
\end{equation}

\textit{Solución}

\begin{equation*}
    \begin{gathered}
        M(x,y)=2x+6y\;\;\;\&\;\;\;N(x,y)=2y-6x\\\\
        \frac{\partial{M}}{\partial{y}}=6\;\;\;\&\;\;\;\frac{\partial{N}}{\partial{x}}=-6\\\\
        \therefore\;\;\frac{\partial{M}}{\partial{y}}\neq\frac{\partial{N}}{\partial{x}}
    \end{gathered}
\end{equation*}

En el siguiente tema se mostrara el como se resulven ED de primer orden que no cumplen con la propiedad de ser una ED exacta.\\
\textbf{Ejercicios propuestos}
\begin{enumerate}
  \item \(\displaystyle (3y^{2}+2y\sin(2x))=\left(\cos(2x)-6xy-4\frac{4}{1+y^{2}}\right)y'\)
  \item \(\displaystyle \left(y\sin(x)+\sin(y)+\frac{1}{x}\right)dx+\left(x\cos(y)-\cos(x)+\frac{1}{y}\right)dy=0\)
\end{enumerate}