\section{Por partes}
\textbf{Definición:} Sea un función dada por el producto de dos funciones cuya función se desea buscar su integral indefinida, la formula esta dada por:\\

\begin{equation*}
    \begin{gathered}
        \int u dv = uv - \int vdu
    \end{gathered}
\end{equation*}

\emph{Algunos tipos a la hora de aplicar la regla de la integración por partes tenemos que tomar encuenta que la aplicación va acorde a la presedencia de sus funciones}
% \(\displaystyle \) %
\begin{itemize}
  \item Recordemos la famosa regla ILATE que nos ayuda a definir la prioridad de  \(\displaystyle u\)
  \item I es aplicado para las funciones inversas
  \item L es aplicado para aquellas funciones logarítmicas
  \item A es aplicado a todas esas funciones algebraicas
  \item T es aplicado a todas las funciones trigonométricas
  \item E es aplicado a todas las funciones que sean exponenciales
\end{itemize}

Dicho esto continuemos\\

\textbf{Ejemplos:}\\
% Ecuación 1 %
% \(\displaystyle \) %
\begin{equation}
    \begin{gathered}
        \int \arcsen(x)dx
    \end{gathered}
\end{equation}
\textit{Solución}\\

Por lo cual aplicamos las reglas y sabemos que hay una función inversa con respecto a la función inversa del \(\displaystyle\sen(x)\) por lo que proponemos a \(\displaystyle u=\arcsen(x)\rightarrow du = \frac{1}{\sqrt{1-x^{2}}}dx\) \& \(\displaystyle dv=dx \rightarrow v=x\)

\begin{equation*}
    \begin{gathered}
        x\arcsen(x)-\int \frac{x}{\sqrt{1-x^{2}}}dx
    \end{gathered}
\end{equation*}

Resolviendo la ultima integral haciendo cambio de variable \(\displaystyle w=1-x^{2} \rightarrow dw=-2x dx\)

\begin{equation*}
    \begin{gathered}
        x\arcsen(x)-\int \frac{1}{\sqrt{w}}dw \Leftrightarrow x\arcsen(x)+\frac{1}{2}\sqrt{w} + C
    \end{gathered}
\end{equation*}

Sustituyendo \(\displaystyle w\) con su valor de \(\displaystyle x\) tenemos:

\begin{equation*}
    \begin{gathered}
        \int \arcsen(x)dx=x\arcsen(x)+\frac{1}{2}\sqrt{1-x^{2}}+C
    \end{gathered}
\end{equation*}

\vspace{1cm}
% Ecuación 2 %
% \(\displaystyle \) %
\begin{equation}
    \begin{gathered}
        \int t^{6}\ln(t)dt
    \end{gathered}
\end{equation}

\textit{Solución}\\

De acuerdo a lo que tenemos aquí y a las reglas sabemos, tendremos lo siguiente \(\displaystyle u=\ln(t) \rightarrow du=\frac{1}{t}dt\) \& \(\displaystyle dv=t^{6}dt \rightarrow v=\frac{t^7}{7}\)

\begin{equation*}
    \begin{gathered}
        \frac{t^7}{7}\ln(t)-\frac{1}{7}\int t^{6}dt
    \end{gathered}
\end{equation*}

Ahora resolviendo la ultima integral tenemos que la resolución es:

\begin{equation*}
    \begin{gathered}
        \int t^{6}\ln(t)dt = \frac{t^{7}}{7}-\frac{t^{7}}{49} + C
    \end{gathered}
\end{equation*}

\vspace{1cm}
% Ecuación 3 %
% \(\displaystyle \) %
\begin{equation}
    \begin{gathered}
        \int t^{n}\ln(t)dt \colon n \in \mathbb{N}
    \end{gathered}
\end{equation}
\textit{Solución}\\
Aplicando la regla ILATE tenemos lo siguiente: \(\displaystyle u=\ln(t)\rightarrow du=\frac{1}{t}dt\) \& \(\displaystyle dv=t^{n}dt \rightarrow v=\frac{t^{n+1}}{n+1}\), ahora aplicando la regla tenemos:

\begin{equation*}
    \begin{gathered}
        \frac{t^{n+1}}{n+1}\ln(t)-\frac{1}{n+1}\int t^{n}dt
    \end{gathered}
\end{equation*}

Ahora solo solucionamos la ultima integral:

\begin{equation*}
    \begin{gathered}
        \int t^{n}\ln(t)dt = \frac{t^{n+1}}{n+1}\ln(t)-\frac{t^{n+1}}{(n+1)^{2}} + C \colon \forall n \in \mathbb{N}
    \end{gathered}
\end{equation*}

\vspace{1cm}
% Ecuación 4 %
% \(\displaystyle \) %
\begin{equation}
    \begin{gathered}
        \int e^{x}\sin(x)dx
    \end{gathered}
\end{equation}

De esta integral partiremos de dos soluciones distintas ambas llegando al mismo resultado:\\

\textit{Solución 1}\\

Tomaremos esta integral de forma directa de tal forma que tendremos lo siguiente \(\displaystyle u=\sin(x)\rightarrow du= \cos(x)dx\) \& \(\displaystyle dv=e^{x}dx \rightarrow v=e^{x}\), aunado a esto tenemos:

\begin{equation*}
    \begin{gathered}
        e^{x}\sin(x) - \int e^{x}cos(x)dx 
    \end{gathered}
\end{equation*}

Volveremos a aplicar la regla de la integral por parte esta vez modificando las letras para hacerlo un poco más explicito \(\displaystyle u=a\) \& \(\displaystyle v=b\), por lo que tendremos es \(\displaystyle a=cos(x) \rightarrow da=-sin(x)dx\) \& \(\displaystyle db=e^{x}dx \rightarrow b=e^{x}\), ahora continuando el desarrollo de la integral tendremos:

\begin{equation*}
    \begin{gathered}
        e^{x}\sin(x)-\{e^{x}\cos(x)+\int e^{x}\sin(x)dx\}
    \end{gathered}
\end{equation*}

Ahora aplicando distribución y signos tenemos:

\begin{equation*}
    \begin{gathered}
        e^{x}\sin(x)-e^{x}\cos(x)-\int e^{x}\sin(x)dx
    \end{gathered}
\end{equation*}

Como podemos ver esto se trata de la misma integral a la que tenemos originalmente, lo que nos permite concluir que se trata de una integral ciclica, por lo tanto lo que haremos es lo siguiente:

\begin{equation*}
    \begin{gathered}
        \int e^{x}\sin(x)dx = e^{x}\sin(x)-e^{x}\cos(x)-\int e^{x}\sin(x)dx
    \end{gathered}
\end{equation*}

Algebraicamente haremos que la integral del lado derecho pase del lado izquierdo:

\begin{equation*}
    \begin{gathered}
        2\int e^{x}\sin(x)dx = e^{x}\sin(x)-e^{x}\cos(x)
    \end{gathered}
\end{equation*}

Ahora por simple algebra:

\begin{equation*}
    \begin{gathered}
        \int e^{x}\sin(x)dx = \frac{1}{2}\{e^{x}\sin(x)-e^{x}\cos(x)\}+C
    \end{gathered}
\end{equation*}

\textit{Solución 2}\\
Veamos a \(\displaystyle\sin(x)\) como su representación en el campo de los números complejos con ayuda de la identidad de Euler:

\begin{equation*}
    \begin{gathered}
        \sin(x)= \frac{1}{2i}(e^{ix}-e^{-ix})
    \end{gathered}
\end{equation*}

Para complementar esto veamos igual al \(\displaystyle\cos(x)\) en su forma de Euler:

\begin{equation*}
    \begin{gathered}
        \cos(x)= \frac{1}{2}(e^{ix}+e^{-ix})
    \end{gathered}
\end{equation*}

Entonces tendremos:

\begin{equation*}
    \begin{gathered}
        \frac{1}{2i}\int e^{x}(e^{ix}-e^{-ix})dx \Leftrightarrow \frac{1}{2i}\{\int e^{x(1+i)} - \int e^{x(1-i)}\}
    \end{gathered}
\end{equation*}

Por lo que propondremos el cambio de variable respectivo \(\displaystyle u\) y \(\displaystyle v\) donde \(\displaystyle u=x(1+i)\) y \(\displaystyle v=x(1-i)\), entonces sus derivadas con respecto a \(\displaystyle x\) son \(\displaystyle du=(1+i)dx\) y \(\displaystyle dv=(1-i)dx\), entonces:

\begin{equation*}
    \begin{gathered}
        \frac{1}{2i}\left\{\frac{1}{1+i}e^{x(1+i)}-\frac{1}{1-i}e^{x(1-i)}\right\}+C
    \end{gathered}
\end{equation*}

Ahora aplicaremos el complemento de cada fracción compleja por lo que en el resultado tendremos:

\begin{equation*}
    \begin{gathered}
        \frac{1}{2i}\left\{\frac{1+i}{2}e^{x(1+i)}-\frac{1-i}{2}e^{x(1-i)}\right\}+C
    \end{gathered}
\end{equation*}

Ahora separando términos del campo \(\displaystyle\mathbb{R}\) y \(\displaystyle\mathbb{C}\) tendremos:

\begin{equation*}
    \begin{gathered}
        \frac{1}{4i}\left\{e^{x(1+i)}-e^{x(1-i)} + i(e^{x(1+i)}+e^{x(1-i)})\right\}+C
    \end{gathered}
\end{equation*}

Separando términos y reordenando:

\begin{equation*}
    \begin{gathered}
        \frac{1}{4i}\left\{e^{x(1+i)}-e^{x(1-i)}\right\} + \frac{1}{4}\left\{(e^{x(1+i)}+e^{x(1-i)})\right\}+C
    \end{gathered}
\end{equation*}

Factorizando todos los valores de \(\displaystyle e^{x}\) tendremos:

\begin{equation*}
    \begin{gathered}
        \frac{e^{x}}{4i}\left\{e^{xi}-e^{-xi}\right\} + \frac{e^{x}}{4}\left\{(e^{xi}+e^{-xi})\right\}+C
    \end{gathered}
\end{equation*}

Ahora aplicando identidad de Euler:

\begin{equation*}
    \begin{gathered}
        \frac{1}{2}e^{x}\sin(x) + \frac{1}{2}e^{x}\cos(x)+C
    \end{gathered}
\end{equation*}

Reordenando todo:

\begin{equation*}
    \begin{gathered}
        \frac{1}{2i}\int e^{x}(e^{ix}-e^{-ix})dx \Leftrightarrow \frac{1}{2i}\left\{\int e^{x(1+i)} - \int e^{x(1-i)}\right\} = \frac{1}{2}e^x\left\{\sin(x)+\cos(x)\right\}+C
    \end{gathered}
\end{equation*}
\vspace{1cm}
% Ecuación 5 %
% \(\displaystyle \) %
\begin{equation}
    \begin{gathered}
        \int \sin^{n}(x)dx \colon \forall n\in\mathbb{N}\geq 2
    \end{gathered}
\end{equation}

\textit{Solución}\\

Reescribiremos la integral como:

\begin{equation*}
    \begin{gathered}
        \int \sin^{n-1}(x)\sin(x)dx
    \end{gathered}
\end{equation*}

Ahora si podremos aplicar las reglas podemos considerar que \(\displaystyle\sin^{n-1}(x)\) como un valor exponencial, por lo que propondremos lo siguiente, \(\displaystyle u=\sin^{n-1}(x) \rightarrow du=(n-1)\sin^{n-2}(x)\cos(x)dx\) \& \(\displaystyle dv = \sin(x)dx \rightarrow v=-\cos(x)\), ahora:

\begin{equation*}
    \begin{gathered}
        -\cos(x)\sin^{n-1}(x) + (n-1)\int \sin^{n-2}(x)\cos^{2}(x)dx
    \end{gathered}
\end{equation*}

Recordemos las identidades trigonométricas de valores que son iguales a \(\displaystyle 1\):

\begin{equation*}
    \begin{gathered}
        1=\cos^2(x)+\sin^{2}(x)\\\\
        1=\sec^{2}(x)-\tan^{2}(x)
    \end{gathered}
\end{equation*}

Aplicaremos esta identidad en la integral de tal forma que podremos sustituir la integral a dos partes:

\begin{equation*}
    \begin{gathered}
        -\cos(x)\sin^{n-1}(x) + (n-1)\left\{\int \sin^{n-2}(x)dx -\int\sin^{n}(x)dx\right\}
    \end{gathered}
\end{equation*}

Aplicando distributividad:

\begin{equation*}
    \begin{gathered}
        \int\sin^{n}(x)dx=-\cos(x)\sin^{n-1}(x) + (n-1)\int \sin^{n-2}(x)dx -(n-1)\int\sin^{n}(x)dx
    \end{gathered}
\end{equation*}

Ahora aplicando algebra:

\begin{equation*}
    \begin{gathered}
        n\int\sin^{n}(x)dx=-\cos(x)\sin^{n-1}(x) + (n-1)\int \sin^{n-2}(x)dx + C
    \end{gathered}
\end{equation*}

De nuevo aplicando algebra:

\begin{equation*}
    \begin{gathered}
        \int\sin^{n}(x)dx=\frac{1}{n}\left\{-\cos(x)\sin^{n-1}(x) + (n-1)\int \sin^{n-2}(x)dx\right\} + C \colon \forall n\in\mathbb{N}\geq 2
    \end{gathered}
\end{equation*}

\vspace{1cm}
% Ecuación 6 %
% \(\displaystyle \) %
\begin{equation}
    \begin{gathered}
        \int \cos^{n}(x)dx \colon \forall n\in\mathbb{N}\geq 2
    \end{gathered}
\end{equation}

\textit{Solución}\\

Ahora ya que tenemos noción de la integral anterior podemos proceder con lo siguiente:

\begin{equation*}
    \begin{gathered}
        \int\cos^{n-1}(x)\cos(x)dx\\\\
        u=\cos^{n-1}(x)\rightarrow du=-\cos^{n-2}(x)\sin(x)dx\\\\
        dv=\cos(x)dx \rightarrow v=\sin(x)
    \end{gathered}
\end{equation*}

Esto implica:

\begin{equation*}
    \begin{gathered}
        \cos^{n-1}(x)\sin(x)+(n-1)\int\cos^{n-2}\sin^{2}(x)dx
    \end{gathered}
\end{equation*}

Aplicando igual identidades trigonométrica directamente:

\begin{equation*}
    \begin{gathered}
        \cos^{n-1}(x)\sin(x)+(n-1)\left\{\int\cos^{n-2}(x)dx - \int\cos^{n}(x)dx\right\}\\\\
        \int\cos^{n}(x)dx =\cos^{n-1}(x)\sin(x)+(n-1)\int\cos^{n-2}(x)dx-(n-1)\int\cos^{n}(x)dx\
    \end{gathered}
\end{equation*}

Ahora con algebra:

\begin{equation*}
    \begin{gathered}
        n\int\cos^{n}(x)dx =\cos^{n-1}(x)\sin(x)+(n-1)\int\cos^{n-2}(x)dx+C\\\\
        \int\cos^{n}(x)dx =\frac{1}{n}\left\{\cos^{n-1}(x)\sin(x)+(n-1)\int\cos^{n-2}(x)dx\right\}+C \colon \forall n\in\mathbb{N}\geq 2
    \end{gathered}
\end{equation*}

\textbf{Ejercicios propuestos}

\begin{enumerate}
  \item \(\displaystyle\int\ln(t)dt\)
  \item \(\displaystyle\int \cos^{3}(x)dx\)
  \item \(\displaystyle\int \sin^{5}(x)dx\)
\end{enumerate}
