\chapter{Técnicas de Integración}
Las técnicas de integración son herramientas que nos seran de utilidad como si fuese la frase celebre o la bendición del día a día.\\
Recordemos que las técnicas más comunes que tenemos son 4:
\begin{enumerate}
  \item Por sustitución
  \item Por partes
  \item Por sustitución trigonométrica
  \item Por fracciones parciales
\end{enumerate}
Ahora con esto, comencemos.
\section{Por sustitución}
\textbf{Teorema:} Sea \(\displaystyle g(x) \) una función derivable y supongase que \(\displaystyle F(x)\) es una antiderivada de \(\displaystyle f(x)\).\\

Entonces si \(\displaystyle u=g(x)\) tenemos lo siguiente:

\begin{equation*}
    \begin{gathered}
        \int f(g(x))g'(x)dx = \int f(u)du = F(u) + C = F(g(x)) + C   \colon C \in \mathbb{R}
    \end{gathered}
\end{equation*}

\textbf{Ejemplos:}
% Ecuacion 1 % 
% \(\displaystyle \) %
\begin{equation}
    \begin{gathered}
        \int \frac{x}{\cos^{2}(x^{2})}dx
    \end{gathered}
\end{equation}

\textit{Solución}\\

Recordemos identidades trigonométrica como lo es: \(\displaystyle \frac{1}{\cos(x)}=\sec(x) \), entonces tenemos lo siguiente

\begin{equation*}
    \begin{gathered}
        \int x \sec^{2}(x^2)dx
    \end{gathered}
\end{equation*}

Ahora por sustitución definimos a \(\displaystyle u=x^{2} \rightarrow du=2x dx \) completando la integral tenemos:

\begin{equation*}
    \begin{gathered}
        \frac{1}{2}\int \sec^{2}(u)du = \frac{1}{2}\tan(u) + C
    \end{gathered}
\end{equation*}

Sustituyendo los valores de \(\displaystyle u\) tenemos la integral resuelta:

\begin{equation*}
    \begin{gathered}
        \int \frac{x}{\cos^{2}(x^{2})} = \frac{1}{2} \tan(x^{2}) + C 
    \end{gathered}
\end{equation*}

\vspace{1cm}
% Ecuacion 2 % 
% \(\displaystyle \) %
\begin{equation}
    \begin{gathered}
        \int \frac{3}{\sqrt{5-9x^{2}}}dx
    \end{gathered}
\end{equation}

\textit{Solución}\\
Recordemos la forma de las integrales que pasan a la forma inversa de una función trigonométrica (arcos) podemos pensar en el cambio de variable por \(\displaystyle u=3x \rightarrow du=3dx\), entonces tendremos lo siguiente:

\begin{equation*}
    \begin{gathered}
        \int \frac{1}{\sqrt{5-u^{2}}}du = \arcsen\left(\frac{u}{5}\right) + C
    \end{gathered}
\end{equation*}
Sustituyendo \(\displaystyle u\) por el valor que tenemos en \(\displaystyle x\) tenemos:\\
\begin{equation*}
    \begin{gathered}
        \int \frac{3}{\sqrt{5-9x^{2}}}dx = \arcsen\left(\frac{3x}{5}\right) + C
    \end{gathered}
\end{equation*}

\vspace{1cm}
% Ecuacion 3 % 
% \(\displaystyle \) %
\begin{equation}
    \begin{gathered}
        \int \frac{6 e^{\frac{1}{x}}}{x^{2}}dx
    \end{gathered}
\end{equation}
\textit{Solución}\\
Para este tipo de integral lo que podemos hacer es proponer el siguiente cambio de variable \(\displaystyle u = \frac{1}{x} \rightarrow du=-\frac{1}{x^{2}}dx\)

\begin{equation*}
    \begin{gathered}
        -6\int e^{u}du = -6 e^{u} + C 
    \end{gathered}
\end{equation*}
Sustituyendo \(\displaystyle u\) por su valor con respecto a \(\displaystyle x\) tenemos:

\begin{equation*}
    \begin{gathered}
        \int \frac{6 e^{\frac{1}{x}}}{x^{2}}dx=-6 e^{\frac{1}{x}} + C
    \end{gathered}
\end{equation*}

\vspace{1cm}
% Ecuacion 4 % 
% \(\displaystyle \) %
\begin{equation}
    \begin{gathered}
        \int\frac{e^{x}}{4 + 9 e^{2x}}dx
    \end{gathered}
\end{equation}
\textit{Solución}\\
Se propone el siguiente cambio de variable \(\displaystyle u=3e^{x}\rightarrow du= 3e^{x}dx\)

\begin{equation*}
    \begin{gathered}
        \frac{1}{3}\int \frac{1}{4+u^{2}}du=\frac{1}{6}\arctan{\frac{u}{2}}+C
    \end{gathered}
\end{equation*}
Sustituyendo los valores de \(\displaystyle u \) con respecto a \(\displaystyle x\)
\begin{equation*}
    \begin{gathered}
        \int\frac{e^{x}}{4 + 9 e^{2x}}dx = \frac{1}{6}\arctan{\frac{3e^{x}}{2}}+C
    \end{gathered}
\end{equation*}
\textbf{Ejecicio propuesto:}\\
\begin{enumerate}
  \item \(\displaystyle\int \frac{a^{\tan(t)}}{\cos^{2}(t)}dt\)
    
  \textbf{Hint:} No te espantes y sustituye ese \(\displaystyle\frac{1}{cos^{2}(x)}\) por su identidad trigonométrica correspodiente.
\end{enumerate}
\clearpage