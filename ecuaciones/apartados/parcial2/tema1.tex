\chapter{Ecuaciones de Orden Superior}

La ED de orden superior cuenta con la siguiente forma:

\begin{equation*}
    \begin{gathered}
        a_{n}(x)\frac{d^{n}x}{dy^{n}}+a_{n-1}(x)\frac{d^{n-1}x}{dy^{n-1}}+\cdots+a_{2}\frac{d^{2}x}{dy^{2}}+a_{1}\frac{dx}{dy}+a_{0}(x)y=g(x)
    \end{gathered}
\end{equation*}

Se dice llamar ED lineal de Orden-n donde los coeficientes \(\displaystyle a_{i}(x)\;\forall i\) y \(\displaystyle g(x)\) son funciones continuas en un intervalo I.\\
La ED se dice homogénea si \(\displaystyle g(x)=0\;\forall x\in I\) de otra forma la ED es no-homogénea \(\displaystyle g(x)\neq 0\)\\

\textbf{Ejemplo}\\

\begin{equation*}
  \begin{gathered}
    \displaystyle 2y''+3y'-6y=0\;\;\;\;\;\;\;\;\;x^{3}y'''+6y'+10y=e^{x}
  \end{gathered}
\end{equation*}

Donde podemos ver que la ED de la izquierda si es homogénea, mientras que la ED de la derecha es no-homogénea.\\

Supongamos que los coeficientos de la ED son números reales, entonces la ED se transforma en:

\begin{equation*}
    \begin{gathered}
        a_{n}y^{n}+a_{n-1}y^{n-1}+\cdots+a_{2}y''+a_{1}y'+a_{0}y=g(x)
    \end{gathered}
\end{equation*}
A esta ecuación se le llama ED lineal de coeficientes constantes.\\

\textbf{Ejemplo}\\

\begin{equation*}
  \begin{gathered}
    \displaystyle y'''+2y''+y'=e^{x}\;\;\;\;\;\;\;\;\;xy''+2y'+xy=0\\\\
    y'''-y''=12x^{2}+6x
  \end{gathered}
\end{equation*}
Donde podemos ver que las dos primeras ecuaciones de la izquierda son ED's con coeficientes constantes, mientras la ED de la derecho es una ED con coeficientes variables.

\clearpage

\section{Problemas de valor inicial}

Para una ED de orden-n sobre un intervalo I el problema de valor inicial consiste en que la ED y las n-condiciones, las cuales son:

\begin{equation*}
    \begin{gathered}
        y(x_{0})=y_{0},\;y'(x_{0})=y_{1},\;y''(x_{0})=y_{2},\;\cdots,\;y^{n-1}(x_{0})=y_{n-1}
    \end{gathered}
\end{equation*}

Para cualquier punto \(\displaystyle x_{0}\in I\) tendremos al Teorema de superposición:\\

\textbf{Teorema: (Teorema de superposición)}\\

Sean \(\displaystyle y_{1},\;y_{2},\;\cdots,\;y_{k}\) soluciones de la ED-homogénea de orden-n en un intervalor I.\\

Entonces \(\displaystyle C_{1}y_{1},\;C_{2}y_{2},\;\cdots,\;C_{k}y_{k}\) también son soluciones de la ED-homogénea, así como la combinación lineal de estas soluciones, es decir:
\begin{equation*}
    \begin{gathered}
        C_{1}y_{1}+C_{2}y_{2}+\cdots+C_{k}y_{k}=y(x)
    \end{gathered}
\end{equation*} Que es la solución de la ED.\\

\textbf{Corolario:}

\begin{enumerate}
  \item Un múltiplo constante \(\displaystyle y=C_{i}y_{i}(x)\) de una solución \(\displaystyle y_{i}(x)\) es una ED lineal homogénea, es también solución
  \item Una ED lineal homogénea posee siempre la solución trivial \(\displaystyle y(x)=0\)
\end{enumerate}

\section{Ecuaciones lineales homogeneas de coeficientes constantes}

Sea dada la ED donde los coeficientes \(\displaystyle a_{i}\forall i\) son constantes reales:

\begin{equation*}
    \begin{gathered}
        a_{n}y^{n}+a_{n-1}y^{n-1}+\cdots+a_{2}y''+a_{1}y'+a_{0}y=0
    \end{gathered}
\end{equation*}

Consideremos la ecuación asociada (o caracteristica) de la forma:

\begin{equation*}
    \begin{gathered}
        a_{n}\lambda^{n}+a_{n-1}\lambda^{n-1}+\cdots+a_{2}\lambda^{2}+a_{1}\lambda+a_{0}=0
    \end{gathered}
\end{equation*}

Donde los \(\displaystyle \lambda\) son raíces o ceros de la ecuación asociada, entre las cuales pueden haber raíces multiples y/o complejas.\\
Por ello:
\begin{enumerate}
  \item Raíces Reales y distintas
  \begin{itemize}
    \item Si \(\displaystyle \lambda_{1},\;\lambda_{2},\;\cdots,\;\lambda_{n}\) son reales y distintas, la solución de la ED-lineal-homogénea es de la forma:
    \item \(\displaystyle y(x)=C_{1}e^{\lambda_{1}x}+C_{2}e^{\lambda_{2}x}+\cdots+C_{n}e^{\lambda_{n}x}\)
  \end{itemize}
  \item Raíces Reales multiples:
  \begin{itemize}
    \item Las raices de la ecuación asociada son reales, pero algunas son iguales, denotados por: \(\displaystyle \lambda_{1}=\lambda_{2}=\cdots=\lambda_{k}=\bar{\lambda}\); de modo que \(\displaystyle \bar{\lambda}\) es una raíz k-multiple de la solución, mientras que las demas (n-k) raíces son distintas, la solución es de la forma:
    \item \(\displaystyle y(x)=C_{1}e^{\bar{\lambda_{1}}x}+C_{2}xe^{\bar{\lambda_{2}}x}+\cdots+C_{k}x^{k-1}e^{\bar{\lambda_{n}}x}+C_{k+1}e^{\lambda_{k+1}x}+\cdots+C_{n}e^{\lambda_{n}x}\)
  \end{itemize}
  \item Raíces Complejas (Imaginarias) y distintas
  \begin{itemize}
    \item Si \(\displaystyle \lambda_{j}=\alpha_{j}\pm i\beta_{j}\) son distintos entre todas las soluciones, es decir  \(\displaystyle\lambda_{1}\neq\lambda_{2}\neq\cdots\neq\lambda_{k}\), las solución quedaria escrita como:
    \item \(\displaystyle y(x)=C_{1}e^{\alpha_{1}x}\cos\left(\beta_{1}x\right)+C_{2}e^{\alpha_{2}x}\sin\left(\beta_{2}x\right) + \cdots + C_{k}e^{\alpha_{k}x}\sin\left(\beta_{k}x\right)+C_{k+1}e^{\lambda_{k+1}x}+\cdots+C_{n}e^{\lambda_{n}x}\)
    \item \textit{Nota:} Recordemos que para este caso: \(\displaystyle e^{i\beta x}=\cos\left(\beta x\right)\) y \(\displaystyle e^{-i\beta x}=\sin\left(\beta x\right)\)
  \end{itemize}
  \item Raíces Complejas multiples:
  \begin{itemize}
    \item Al igual que las raíces reales multiples estas se acompañan por un valor un valor multiplicativo \(\displaystyle x\) por cada raiz repetida.
  \end{itemize}
\end{enumerate}
\textit{Nota:} para resolver y encontrar raíces de cualquier tipo es necesario estudiar el tema de División sintetica.
