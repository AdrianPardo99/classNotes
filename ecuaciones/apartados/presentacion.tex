\chapter{Introducción}
\section{Presentación}
Las ecuaciones diferenciales no son más que un conjunto de sistemas lineales o no lineales que representan modelo de comportamiento matemático, el cual puede ser aplicado en áreas como la ingeniería la cual puede servir para describir el comportamiento físico de un circuito electríco mediante las ecuaciones de voltaje de elementos lineales (resistencias, capacitadores e inductores), por otro lado igual puede ser aplicado en modelados matemáticos que puedan representar la medición estadística de una población de bacterias.\\
Ahora bien con el fin de facilitar el aprendizaje de estos temas se desarrollaran varias notas que se tomaron en el curso así como complemento de propía mano del escritor.
\section{Prerequisitos para la materia}
Si bien las matemáticas guardan una intima pero fuerte relación entre sus topicos es importante marcar algunos prerequisitos para poder dar seguimiento o poder tener un mejor entendimiento con la materia
\begin{center}
  \begin{tabular}{|p{5.5cm}|p{5.5cm}|}
    \hline
    Principios de Cálculo & Nociones de Algebra Lineal\\
    \hline
    Nociones de Análisis Vectorial & Nociones de Física\\
    \hline
  \end{tabular}
\end{center}
Si bien podriamos desglosar todos y cada uno de los prerequisitos de cada asignatura, no es la idea asustar a los pequeños lectores o consultores de este documento, ahora si continuemos

\section{Libros de consulta}
Si bien las ecuaciones diferenciales pueden ser estudiadas en cursos los cuales no soliciten libro, te puedes apoyar de material que se encuentra de forma gratuita pero quizas poco legal en internet, por ello yo no quiero alentarte a realizar una conducta dañina a los autores, mi mejor recomendación en este aspecto, es quizas consigue los pdf's pero después de ello busca la forma de conseguirlo en físico para tu formación o para el apoyo de tus compañeros o alumnos.
\begin{itemize}
  \item Dennis Zill, Ecuaciones Diferenciales \(\displaystyle \rightarrow\) Para comenzar desde el inicio y sin complicaciones
  \item Editorial Trillas Canek, Ecuaciones Diferenciales ordinarias \(\displaystyle \rightarrow\) Para subir el nivel con respecto al Dennis
  \item Makarenko, Problemas de Ecuaciones Diferenciales ordinarias 1996 \(\displaystyle \rightarrow\) Para ejercicios bastante completos y extensos
\end{itemize}
\section{Inicio}
Recordemos que para tener un buen inicio con respecto al curso es necesario tener un ligero repaso a las Técnicas de Integración y a algunas técnicas de identificación de identidades trigonométricas, por lo que durante el desarrollo de este texto se añadieron varios formularios respecto al tema.